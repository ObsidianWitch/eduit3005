\documentclass[frenchb, 11pt]{article}

\usepackage[top=2cm, bottom=2cm, left=2cm, right=2cm]{geometry}
\usepackage[utf8]{inputenc}
\usepackage[T1]{fontenc}
\usepackage{babel}

\usepackage{multicol}
\setlength{\columnseprule}{1pt} % separation line between columns

\usepackage{hyperref}
\hypersetup{
	colorlinks=true,	% false: boxed links; true: colored links
	linkcolor=black,	% color of internal links
	urlcolor=blue,		% color of external links
	citecolor=blue
}

\usepackage{graphicx}	% import graphics
\usepackage{wrapfig}	% wrap text around figures
\usepackage{subcaption}


\usepackage{verbatim}	% multi-line comments

% Colors
\usepackage[usenames,dvipsnames]{xcolor}

% Colored frame
\usepackage{mdframed}
\usepackage{framed}
\definecolor{shadecolor}{rgb}{0.96,0.96,0.96}
\definecolor{TFFrameColor}{rgb}{0.96,0.96,0.96}
\definecolor{TFTitleColor}{rgb}{0.00,0.00,0.00}

% Redefine leftbar envvironment
\newlength{\leftbarwidth}
\setlength{\leftbarwidth}{1pt}
\newlength{\leftbarsep}
\setlength{\leftbarsep}{10pt}

\newcommand*{\leftbarcolorcmd}{\color{leftbarcolor}} % as a command to be more flexible
\colorlet{leftbarcolor}{gray}

\renewenvironment{leftbar}{%
    \def\FrameCommand{{\leftbarcolorcmd{\vrule width \leftbarwidth\relax\hspace {\leftbarsep}}}}%
    \MakeFramed {\advance \hsize -\width \FrameRestore }%
}{%
    \endMakeFramed
}

% Code listings
\usepackage{listings}
\definecolor{dkgreen}{rgb}{0,0.6,0}
\definecolor{gray}{rgb}{0.5,0.5,0.5}
\definecolor{mauve}{rgb}{0.58,0,0.82}
\definecolor{blue}{rgb}{0,0,0.7}
\definecolor{lightred}{rgb}{1,0.96,0.96}
\definecolor{darkred}{rgb}{0.85,0.33,0.31}
\lstset{
	language=bash,
	basicstyle=\scriptsize,
	numbers=left,                   % where to put the line-numbers
  	numberstyle=\tiny\color{gray},
	commentstyle=\color{dkgreen},
	stringstyle=\color{BrickRed},
	backgroundcolor=\color{shadecolor},
    keywordstyle=\color{OliveGreen},
	frame=single,                   % adds a frame around the code
 	rulecolor=\color{black},
	emph={},
	emphstyle=\color{mauve},
	morekeywords={},
	keywordstyle={\color{black}},
	showstringspaces=false,
  	tabsize=4,
	moredelim=[is][\small\ttfamily]{/*}{*/},
	breaklines=true
}

% Title page
\title{
	\textbf{IT-3005 - TP Sécurité}\\
	DMZ / Firewall Linux
}
\date{\today}

\begin{document}
\maketitle % TODO lien vers archive contenant tous les fichiers du TP
\newpage

\tableofcontents
\newpage

\section{Introduction}
% postes utilisés 10,11,12
% objectifs
% brève présentation du lab créé à l'aide de Netkit

\section{Câblage}
Lors de la séance nous utilisions les postes 10 (LAN), 11 (Routeur) et 12 (DMZ). La première étape du TP consistait à câbler (câbles RJ45 croisés) la machine LAN au routeur, et la DMZ au routeur. Nous avons tout d'abord câblé \textbf{A10 (LAN)} à \textbf{A11 (Routeur)}, puis avons executé \emph{mii-tool} ce qui nous a permis de déterminer que \textbf{A10} correspondait à \textbf{eth2} sur la machine LAN et que \textbf{A11} correspondait à \textbf{eth2} sur le routeur. Nous avons ensuite branché \textbf{B11 (Routeur)} à \textbf{A12 (DMZ)} et identifié avec \emph{mii-tool} que \emph{B11} correspondait à \textbf{eth1} sur le routeur, et \textbf{A12} correspondait à \textbf{eth1} sur la machine DMZ.\\

Souhaitant réaliser de nouveau le TP chez nous, nous avons créé un lab pour Netkit. Le "câblage" de deux machines est décrit par des domaines de collision. Le listing ci-dessous contient la configuration du lab, celle-ci correspond à la figure \ref{fig:planadrnetkit}.
\begin{lstlisting}[caption=lab.conf]
LAB_DESCRIPTION="TP securite - DMZ/Firewall"

LAN[0]=tap,10.0.0.1,10.0.0.4
LAN[1]=A

Router[0]=tap,10.0.0.1,10.0.0.2
Router[1]=A
Router[2]=B

DMZ[0]=tap,10.0.0.1,10.0.0.3
DMZ[1]=B
\end{lstlisting}
\textbf{N.B.} Dans le dossier contenant le fichier de configuration du lab, un fichier resolv.conf permettant de spécifier les serveurs DNS à utiliser a été placé dans Router/etc. Lors du lancement de la VM Router, le fichier /etc/resolv.conf par défaut sera remplacé par celui décrit précédemment.\\

Les machines virtuelles associées au lab décrites par le fichier de configuration ci-dessus peuvent être lancées à l'aide de la commande \emph{lstart}. Les interfaces \emph{tap} permettent de partager la connexion Internet de la machine hôte avec une VM. Si des interfaces tap sont présentes, lors de l'exécution de \emph{lstart} une interfacce sera ajoutée sur le PC hôte ainsi qu'une règle nat iptable activant l'IP masquerading (voir listing ci-dessous).

\begin{lstlisting}
obside@obsideb:~$ ifconfig
...
nk_tap_obside Link encap:Ethernet  HWaddr ca:f2:21:69:b3:97  
          inet addr:10.0.0.1  Bcast:10.255.255.255  Mask:255.0.0.0
          inet6 addr: fe80::c8f2:21ff:fe69:b397/64 Scope:Link
          UP BROADCAST RUNNING MULTICAST  MTU:1500  Metric:1
          RX packets:66 errors:0 dropped:0 overruns:0 frame:0
          TX packets:84 errors:0 dropped:0 overruns:0 carrier:0
          collisions:0 txqueuelen:500 
          RX bytes:5032 (4.9 KiB)  TX bytes:10164 (9.9 KiB)
...

obside@obsideb:~$ sudo iptables -L -t nat
Chain PREROUTING (policy ACCEPT)
target     prot opt source               destination         

Chain INPUT (policy ACCEPT)
target     prot opt source               destination         

Chain OUTPUT (policy ACCEPT)
target     prot opt source               destination         

Chain POSTROUTING (policy ACCEPT)
target     prot opt source               destination         
MASQUERADE  all  --  anywhere             anywhere  
\end{lstlisting}
Une fois les VMs éteintes, la commande \emph{vclean --clean-all} permet de retirer les modifications apportées au PC hôte.

\begin{figure}[h!]
	\centering
	\begin{subfigure}[b]{0.4\textwidth}
		\includegraphics[scale=0.80]{sch1.png}
		\caption{Séance}
	\end{subfigure}%
	~
	\begin{subfigure}[b]{0.4\textwidth}
	\includegraphics[scale=0.80]{sch2.png}
	\caption{Netkit}
	\label{fig:planadrnetkit}
	\end{subfigure}
	\caption{Plan d'adressage}
\end{figure}

\begin{mdframed}[backgroundcolor=lightred, linecolor=darkred]
Le rapport décrira les différentes étapes du TP en suivant le plan d'adressage de la figure \ref{fig:planadrnetkit} (Netkit).
\end{mdframed}

% après câblage ping possibles: lan <-> routeur et dmz <-> routeur
% il faut activer le routage IP sur le routeur

\newpage

\section{Routage classique}
Nous avons tout d'abord configuré les interfaces Ethernet des 3 machines, ajouté les passerelles par défaut pour LAN et DMZ, et activé le routage IP (IP forwarding) pour le routeur. Le routage IP permet de prendre des décisions concernant le chemin qu'un paquet doit prendre entre les différents réseaux.

\subsection{Configuration du PC Routeur}
% TODO coloration syntaxique
\begin{lstlisting}
/*Router:~#*/ echo "1" > /proc/sys/net/ipv4/ip_forward
/*Router:~#*/ ifconfig eth1 192.168.1.2 netmask 255.255.255.0 up
/*Router:~#*/ ifconfig eth2 192.168.2.2 netmask 255.255.255.0 up

/*Router:~#*/ ifconfig
eth0      Link encap:Ethernet  HWaddr 76:f5:98:50:22:50
          inet addr:10.0.0.2  Bcast:10.255.255.255  Mask:255.0.0.0
          ...

eth1      Link encap:Ethernet  HWaddr 7e:a1:77:92:58:e8
          inet addr:192.168.1.2  Bcast:192.168.1.255  Mask:255.255.255.0
          ...

eth2      Link encap:Ethernet  HWaddr 0e:bd:8c:c5:14:02
          inet addr:192.168.2.2  Bcast:192.168.2.255  Mask:255.255.255.0
          ...

/*Router:~#*/ route -n
Kernel IP routing table
Destination     Gateway         Genmask         Flags Metric Ref    Use Iface
192.168.2.0     0.0.0.0         255.255.255.0   U     0      0        0 eth2
192.168.1.0     0.0.0.0         255.255.255.0   U     0      0        0 eth1
10.0.0.0        0.0.0.0         255.0.0.0       U     0      0        0 eth0
0.0.0.0         10.0.0.1        0.0.0.0         UG    0      0        0 eth0
\end{lstlisting}
\hfill

\begin{leftbar}
\noindent Q1. Le répertoire \textbf{/proc/sys} contient des fichiers et sous-répertoires permettant d'activer ou désactiver immédiatement des fonctionnalités du kernel. Le répertoire \textbf{/proc/sys/net} concerne des fonctionnalités réseau, et \textbf{/proc/sys/net/ipv4} concerne plus particulièrement les fonctionnalités liées au protocole IPv4. Par exemple, le fichier \textbf{/proc/sys/net/ipv4/ip\_forward} permet d'activer le routage. Le fichier \textbf{/proc/sys/net/ipv4/icmp\_echo\_ignore\_all} permet d'ignorer les requêtes ECHO du protocole ICMP, ainsi la commande \emph{ping} utilisant ce protocole voit ses requêtes ignorées par une machine ayant \emph{icmp\_echo\_ignore\_all} activé.
% TODO serveurs actifs netstat
\end{leftbar}

\subsection{Configuration du PC LAN}
\begin{lstlisting}
/*LAN:~#*/ ifconfig eth0 down
/*LAN:~#*/ ifconfig eth1 192.168.1.1 netmask 255.255.255.0 up
/*LAN:~#*/ route add default gw 192.168.1.2 eth1

/*LAN:~#*/ ifconfig
eth1      Link encap:Ethernet  HWaddr 42:a7:57:d3:a5:1b
          inet addr:192.168.1.1  Bcast:192.168.1.255  Mask:255.255.255.0
          ...

/*LAN:~#*/ route
Kernel IP routing table
Destination     Gateway         Genmask         Flags Metric Ref    Use Iface
192.168.1.0     *               255.255.255.0   U     0      0        0 eth1
default         192.168.1.2     0.0.0.0         UG    0      0        0 eth1

/*LAN:~#*/ route -n
Kernel IP routing table
Destination     Gateway         Genmask         Flags Metric Ref    Use Iface
192.168.1.0     0.0.0.0         255.255.255.0   U     0      0        0 eth1
0.0.0.0         192.168.1.2     0.0.0.0         UG    0      0        0 eth1
\end{lstlisting}

\subsection{Configuration du PC DMZ}
\begin{lstlisting}
/*DMZ:~#*/ ifconfig eth0 down
/*DMZ:~#*/ ifconfig eth1 192.168.2.1 netmask 255.255.255.0 up
/*DMZ:~#*/ route add default gw 192.168.2.2 eth1

/*DMZ:~#*/ ifconfig
eth1      Link encap:Ethernet  HWaddr 86:e8:ec:bc:60:67
          inet addr:192.168.2.1  Bcast:192.168.2.255  Mask:255.255.255.0
          ...
\end{lstlisting}
\newpage

\subsection{Tables de routage}
\begin{leftbar}
	\noindent Q2. La commande \emph{route -n} permet d'afficher la table de routage IP en remplaçant les noms d'hôtes (provenant par exemple du fichier /etc/hosts) par des adresses numériques. Le symbole * est remplacé par \emph{0.0.0.0}, ce qui désigne la route par défaut. Les tables de routage sur les machines LAN, DMZ et Routeur sont les suivantes :
	\begin{lstlisting}[caption=Table de routage LAN, numbers=none]
Kernel IP routing table
Destination     Gateway         Genmask         Flags Metric Ref    Use Iface
192.168.1.0     0.0.0.0         255.255.255.0   U     0      0        0 eth1
0.0.0.0         192.168.1.2     0.0.0.0         UG    0      0        0 eth1
	\end{lstlisting}
	\begin{lstlisting}[caption=Table de routage DMZ, numbers=none]
Kernel IP routing table
Destination     Gateway         Genmask         Flags Metric Ref    Use Iface
192.168.2.0     0.0.0.0         255.255.255.0   U     0      0        0 eth1
0.0.0.0         192.168.2.2     0.0.0.0         UG    0      0        0 eth1
	\end{lstlisting}
	\begin{lstlisting}[caption=Table de routage Routeur, numbers=none]
Kernel IP routing table
Destination     Gateway         Genmask         Flags Metric Ref    Use Iface
192.168.2.0     0.0.0.0         255.255.255.0   U     0      0        0 eth2
192.168.1.0     0.0.0.0         255.255.255.0   U     0      0        0 eth1
10.0.0.0        0.0.0.0         255.0.0.0       U     0      0        0 eth0
0.0.0.0         10.0.0.1        0.0.0.0         UG    0      0        0 eth0
	\end{lstlisting}
\end{leftbar}

\noindent Suite aux configurations effectuées précédemment, il est possible d'effectuer des pings entre les 3 machines.

\subsection{Configuration du serveur web sur le PC DMZ}
Nous souhaitons ensuite modifier la configuration du serveur http Apache (httpd) pour remplacer le port d'écoute par défaut 80 par le port 8080. Pour cela nous avons modifié le fichier suivant.
\begin{lstlisting}[caption=/etc/httpd/conf/httpd.conf]
Listen 8080
\end{lstlisting}
\hfill

\noindent \textbf{N.B.} Sous une machine (Debian testing) créée à l'aide de netkit utilisant le package apache2 et une configuration par défaut, nous avons dû modifier les fichiers \textbf{/etc/apache2/ports.conf} et \textbf{/etc/apache2/sites-enabled/000-default}. Le premier fichier nous a permis de remplacer le port d'écoute par défaut, le second de modifier le port du VirtualHost (les serveurs virtuels permettent de faire fonctionner plusieurs serveurs web sur une même machine, ici nous n'avons qu'un seul VirtualHost que nous souhaitons associé au port 8080).

\begin{lstlisting}[caption=/etc/apache2/ports.conf, escapechar=@]
# If you just change the port or add more ports here, you will likely also
# have to change the VirtualHost statement in
# /etc/apache2/sites-enabled/000-default
# This is also true if you have upgraded from before 2.2.9-3 (i.e. from
# Debian etch). See /usr/share/doc/apache2.2-common/NEWS.Debian.gz and
# README.Debian.gz

@\textbf{NameVirtualHost *:8080}@
@\textbf{Listen 8080}@

<IfModule mod_ssl.c>
    # SSL name based virtual hosts are not yet supported, therefore no
    # NameVirtualHost statement here
    Listen 443
</IfModule>
\end{lstlisting}

\begin{lstlisting}[caption=/etc/apache2/sites-enabled/000-default, escapechar=\%]
%\textbf{<VirtualHost *:8080>}%
        ServerAdmin webmaster@localhost

        DocumentRoot /var/www/
        <Directory />
                Options FollowSymLinks
                AllowOverride None
        </Directory>
		...
\end{lstlisting}

\noindent Une fois les fichiers de configuration modifiés, il faut relancer le serveur web :
\begin{lstlisting}
/etc/init.d/httpd restart

OU (en fonction de la distribution)

/etc/init.d/apache2 restart
\end{lstlisting}
\hfill

\noindent On peut ensuite vérifier à l'aide de la commande ps que le serveur est bien lancé :
\begin{lstlisting}
/*DMZ:~#*/ ps aux | grep apache
root      1010  0.0  8.4  13136  2736 ?        Ss   20:47   0:00 /usr/sbin/apache2 -k start
www-data  1011  0.0  5.9  12908  1928 ?        S    20:47   0:00 /usr/sbin/apache2 -k start
www-data  1013  0.0 10.3 234696  3352 ?        Sl   20:47   0:00 /usr/sbin/apache2 -k start
www-data  1018  0.0 10.3 234696  3340 ?        Sl   20:47   0:00 /usr/sbin/apache2 -k start
root      1116  0.0  2.1   3476   680 tty0     S+   23:45   0:00 grep --color apache
\end{lstlisting}
\hfill

Ci-dessous deux captures d'écran (figures \ref{fig:sch3lan} et \ref{fig:sch3dmz}) de la machine LAN accédant à \textbf{192.168.2.1} (DMZ) sur le port 8080 avec le navigateur web Lynx, et de la machine DMZ observant les paquets reçus sur son interface eth1. Le routeur achemine les paquets de la machine LAN vers la machine DMZ (192.168.1.1:59904 > 192.168.2.1:8080), puis la machine DMZ répond aux requêtes de la machine LAN en envoyant les paquets sur l'interface du routeur à laquelle elle est connectée (192.168.2.1:8080 > 192.168.1.1:59904).

\begin{figure}[h!]
	\centering
	\includegraphics[scale=0.62]{sch3LAN.png}
	\caption{Accès au serveur web (DMZ) depuis Lynx (LAN)}
	\label{fig:sch3lan}
\end{figure}
\newpage

\begin{figure}[h!]
	\centering
	\includegraphics[scale=0.68]{sch3DMZ.png}
	\caption{Paquets reçus par le serveur web (DMZ)}
	\label{fig:sch3dmz}
\end{figure}
\newpage

\section{Mise en oeuvre de la plate-forme sécurisée}
Dans cette partie, nous allons établir des règles de filtrage (table \textbf{filter}) et de translation d'adresse (table \textbf{nat}) à l'aide d'iptables sur le routeur. Pour cela, nous avons créé un script firewall (\emph{/root/firewall}) contenant ces règles.

\begin{leftbar}
	\noindent Q3. Nous souhaitons tout d'abord supprimer les règles établies précédemment, puis rejeter par défaut tous les paquets (INPUT, OUTPUT et FORWARD). Nous établirons par la suite des règles permettant d'autoriser des paquets répondant à certains critères.
	\begin{lstlisting}[numbers=none]
# Flush filter and nat tables
iptables -F
iptables -F -t nat

# Policy : drop all
iptables -P INPUT DROP
iptables -P OUTPUT DROP
iptables -P FORWARD DROP
	\end{lstlisting}
\end{leftbar}

La politique par défaut pour les 3 chaînes (INPUT, OUTPUT, FORWARD) et de rejeter les paquets, il n'est donc entre autres plus possible de ping les 3 PCs entre eux.

\begin{leftbar}
	\noindent Q4. Après exécution du script, les règles d'iptables peuvent être observées à l'aide de la commande \emph{iptables -L}. Nous pouvons observer que la politique par défaut est effectivement le rejet de paquet pour chaque chaîne (e.g. \emph{Chain INPUT (policy DROP)}).
	\begin{lstlisting}[numbers=none]
/*Router:~#*/ iptables -L
Chain INPUT (policy DROP)
target     prot opt source               destination         

Chain FORWARD (policy DROP)
target     prot opt source               destination         

Chain OUTPUT (policy DROP)
target     prot opt source               destination
	\end{lstlisting}
\end{leftbar}

\subsection{Filtrage entre LAN et DMZ}
Nous souhaitons cacher l'existence des machines du réseau LAN aux machines de la DMZ. Avant cela, nous allons établir des règles dans le firewall pour autoriser certains paquets.

\begin{leftbar}
	\noindent Q5. Nous allons tout d'abord autoriser le réseau LAN à envoyer des requêtes du protocole ICMP, puis le routeur à répndre à ces requêtes. Cela va permettre d'accepter un ping venant du réseau LAN (\textbf{192.168.1.0/24}) adressé à l'interface du Routeur côté LAN (\textbf{192.168.1.2}).
	\begin{lstlisting}[numbers=none]
# accept ping LAN -> routeur
iptables -A INPUT -s 192.168.1.0/24 -d 192.168.1.2 -p icmp --icmp-type echo-request -j ACCEPT
iptables -A OUTPUT -s 192.168.1.2 -d 192.168.1.0/24 -p icmp --icmp-type echo-reply -j ACCEPT
	\end{lstlisting}
	\hfill
	
	\noindent Une alternative possible est d'autoriser les requêtes du protocole ICMP provenant du réseau LAN (\textbf{192.168.1.0/24}) adressées à l'interface \textbf{eth1} du routeur (reliée au réseau LAN), puis d'autoriser les réponses du protocole ICMP sortant du routeur par l'interface \textbf{eth1} en direction du réseau LAN.
	\begin{lstlisting}[numbers=none]
# alternative
iptables -A INPUT -s 192.168.1.0/24 -i eth1 -p icmp --icmp-type echo-request -j ACCEPT
iptables -A OUTPUT -o eth1 -d 192.168.1.0/24 -p icmp --icmp-type echo-reply -j ACCEPT
	\end{lstlisting}
	\hfill
	
	\noindent Une autre alternative possible pour la sortie du routeur est d'autoriser les paquets du protocole ICMP ayant déjà une connexion établie. Nous pouvons aussi par exemple retirer les options -s et/ou -d pour réduire les conditions devant être remplies pour qu'une réponse soit acceptée.
	\begin{lstlisting}[numbers=none]
iptables -A OUTPUT -s 192.168.1.2 -d 192.168.1.0/24 -p icmp -m state --state ESTABLISHED,RELATED -j ACCEPT
	\end{lstlisting}
\end{leftbar}

% TODO 3.2 désactiver eth0 lan
% TODO 3.3 désactiver eth0 DMZ

\newpage

% TODO bibliography
\begin{thebibliography}{5}
\end{thebibliography}

\end{document}
